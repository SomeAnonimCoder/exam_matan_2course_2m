
\subsection{Некоторые определения}
Говорят, что функция \(f\left( x \right),\) определенная в интервале \(\left[ {a,b} \right],\) является кусочно непрерывной, если она непрерывна всюду в данном интервале, за исключением конечного числа точек разрыва.
Функция \(f\left( x \right),\) определенная в интервале \(\left[ {a,b} \right],\) является кусочно гладкой, если сама функция и ее производная кусочно непрерывны в заданном интервале.
Частичные суммы ряда Фурье
Введем понятие частичной суммы ряда Фурье \({f_N}\left( x \right)\) функции \(f\left( x \right),\) заданной в интервале \(\left[ {-\pi, \pi} \right].\) Она определяется выражением \[{f_N}\left( x \right) = \frac{{{a_0}}}{2} + \sum\limits_{n = 1}^N {\left( {{a_n}\cos nx + {b_n}\sin nx} \right)} .\] В комплексной форме частичная сумма \({f_N}\left( x \right)\) функции \(f\left( x \right),\) заданной в интервале \(\left[ {-\pi, \pi} \right],\) выражается формулой \[ {{f_N}\left( x \right) = \sum\limits_{n = - N}^N {{c_n}{e^{inx}}} } = {\int\limits_{ - \pi }^\pi {\left( {\frac{1}{{2\pi }}\sum\limits_{n = - N}^N {{e^{in\left( {x - y} \right)}}} } \right)f\left( y \right)dy} .} \]
\subsection{Ядро Дирихле}
Функция \[{D_N}\left( x \right) = \sum\limits_{n = - N}^N {{e^{inx}}} = \frac{{\sin \left( {N + \frac{1}{2}} \right)x}}{{\sin \frac{x}{2}}}\] называется ядром Дирихле.

Частичная сумма ряда Фурье выражается через ядро Дирихле следующим образом: \[ {{f_N}\left( x \right) = \frac{1}{{2\pi }}\int\limits_{ - \pi }^\pi {{D_N}\left( {x - y} \right)f\left( y \right)dy} } = {\frac{1}{{2\pi }}\int\limits_{ - \pi }^\pi {{D_N}\left( y \right)f\left( {x - y} \right)dy} .} \] В данной секции мы рассмотрим три типа сходимости рядов Фурье: сходимость в точке, равномерную сходимость и сходимость в пространстве \({L_2}.\)
\subsection{Сходимость ряда Фурье в точке}
Пусть \(f\left( x \right)\) является кусочно гладкой функцией в интервале \(\left[ {-\pi, \pi} \right].\) Тогда для любого \({x_0} \in \left[ { - \pi ,\pi } \right]\) выполняется условие \[ \lim\limits_{N \to \infty } {f_N}\left( {{x_0}} \right) = \begin{cases} f\left( {{x_0}} \right), & \text{если}\,f\left( x \right)\,\text{непрерывна в}\, \left[ { - \pi ,\pi } \right] \\ \frac{{f\left( {{x_0} - 0} \right) + f\left( {{x_0} + 0} \right)}}{2}, & \text{если}\,f\left( x \right)\,\text{имеет разрыв при}\, {{x_0}} \end{cases}, \] где \({f\left( {{x_0} - 0} \right)}\) и \({f\left( {{x_0} + 0} \right)}\) представляют собой, соответственно, левосторонний и правосторонний пределы в точке \({x_0}.\)

\subsection{Равномерная сходимость ряда Фурье}

Говорят, что последовательность частичных сумм ряда Фурье \(\left\{ {{f_N}\left( x \right)} \right\}\) сходится равномерно к функции \(f\left( x \right),\) если скорость сходимости частичных сумм \({{f_N}\left( x \right)}\) не зависит от \(x.\). Будем говорить, что ряд Фурье функции \(f\left( x \right)\) сходится равномерно к этой функции, если \[\lim\limits_{N \to \infty } \left[ {\max\limits_{x \in \left[ { - \pi ,\pi } \right]} \left| {f\left( x \right) - {f_N}\left( x \right)} \right|} \right] = 0.\] Теорема. Ряд Фурье \(2\pi\)-периодической непрерывной и кусочно гладкой функции сходится равномерно.

\subsection{Сходимость ряда Фурье в пространстве \({L_2}\)}
Пространство \({L_2}\left( { - \pi ,\pi } \right)\) образовано функциями, удовлетворяющими условию \[\int\limits_{ - \pi }^\pi {{{\left| {f\left( x \right)} \right|}^2}dx} < \infty .\] Будем говорить, что функция \(f\left( x \right)\) является квадратично интегрируемой, если она принадлежит классу \({L_2}.\) Если \(f\left( x \right)\) квадратично интегрируема, то \[\lim\limits_{N \to \infty } \frac{1}{{2\pi }}\int\limits_{ - \pi }^\pi {{{\left| {f\left( x \right) - {f_N}\left( x \right)} \right|}^2}dx} = 0,\] то есть частичные суммы \({f_N}\left( x \right)\) сходятся к \(f\left( x \right)\) в смысле среднего квадратичного.

Из равномерной сходимости ряда Фурье следует как сходимость в точке, так и сходимость в пространстве \({L_2}.\) Обратное утверждение неверно: сходимость в пространстве \({L_2}\) не означает, что ряд Фурье сходится в точке или равномерно, и, аналогично, из сходимости в точке не вытекает равномерная сходимость или сходимость в пространстве \({L_2}.\)

