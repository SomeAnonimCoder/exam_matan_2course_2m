

Если $\lim\limits_{n\to\infty}S_n = \lim\limits_{max|\Delta r_i|\to 0}\sum\limits_{i=1}^{n}(a(P_i)\Delta r_i) $ существует и не зависит от способа разбиения дуги $AB$ на участки $P_i$, то предел называется \textbf{линейным интегралом} $\overrightarrow{a}$ по дуге $AB$ в направлении от $A$ до $B$

$$\int_{AB} (\overrightarrow{a}d\overrightarrow{r}) = \lim\limits_{max|\Delta r_i|\to 0}\sum\limits_{i=1}^{n}(a(P_i)\Delta r_i)$$

В координатах:

$$\int_{AB}(\overrightarrow{a}d\overrightarrow{r}) = \int_{AB}a_xdx + a_ydy+a_zdz$$

\textbf{Свойства}

1) Линейность:

$$\int_{AB} ((\lambda\overrightarrow{a} + \mu\overrightarrow{b})d\overrightarrow{r}) = \lambda\int_{AB}(\overrightarrow{a}d\overrightarrow{r})+ \mu\int_{AB}(\overrightarrow{b}d\overrightarrow{r})$$

2) Аддитивность

$$\int_{AB}  (\overrightarrow{a}d\overrightarrow{r}) = \int_{AC}  (\overrightarrow{a}d\overrightarrow{r}) + \int_{CB}  (\overrightarrow{a}d\overrightarrow{r})$$

3) При изменении направления вдоль дуги интеграл меняет знак

$$\int_{AB}  (\overrightarrow{a}d\overrightarrow{r}) = - \int_{AB}  (\overrightarrow{a}d\overrightarrow{r})$$

\textbf{Физический смысл:}

$$\int_{AB} (\overrightarrow{a}d\overrightarrow{r})$$

$$\overrightarrow{a} = \overrightarrow{F}$$
$$\int_{AB} = (\overrightarrow{F}d\overrightarrow{r}) = A$$

Работа силы F по пермещенению по дуге $AB$