\subsection{Соленоидальное поле}

Векторное поле называется соленоидальным или вихревым, если через любую замкнутую поверхность $S$ его поток равен нулю:

$\int\limits_S \vec a \cdot \vec{ds} = 0$.
Другое определение соленоидального поля: Векторное поле $\vec {a}$ называют соленоидальным, если оно является вихрем некоторого поля $\vec  {b}$, $\vec {a}=\mathrm {rot} \,{\vec {b}}$. При этом векторное поле $\vec  {b}$ называют векторным потенциалом поля $\vec {a}$.

Если это условие выполняется для любых замкнутых S в некоторой области (по умолчанию - всюду), то это условие равносильно тому, что равна нулю дивергенция векторного поля $\vec a$:

$\mathrm {div} \,{\vec {a}}\equiv \nabla \cdot {\vec {a}}=0$
всюду на этой области (подразумевается, что дивергенция всюду на этой области существует). Поэтому соленоидальные поля называют также бездивергентными.

Для широкого класса областей это условие выполняется тогда и только тогда, когда $\vec a$ имеет векторный потенциал, то есть существует некое такое векторное поле $\vec  A$ (векторный потенциал), что $\vec a$ может быть выражено как его ротор:

$\vec {a}=\nabla \times {\vec {A}}\equiv \mathrm {rot} \,{\vec {A}}$.

\subsection{Свойства соленоидального поля}

\begin{itemize}
\item Поток соленоидального векторного поля через поверхность $\sigma$, ограничивающую область $V_{\sigma} \in V$, равен нулю. Это прямое следствие формулы Остроградского.
\item Верно и обратное утверждение: равенство нулю потока через любую замкнутую поверхность $\sigma$ достаточно для соленоидальности поля $\bar a(M)$.

\item Пусть в  $V$  имеется изолированный источник или сток поля. Если поле $\bar a(M)$ соленоидально, то его поток через любую замкнутую поверхность $\sigma$, содержащую этот источник имеет одно и то же значение. Фраза "в  $V$  имеется изолированный источник или сток поля" означает, что область  $V$ , в которой поле соленоидально, неодносвязна - из $V$  выколота точка, в которой находится источник. Так, поле электрической напряжённости, создаваемое зарядом  $q$ ,$\bar E=\frac{q}{r^3}\bar r$, соленоидально всюду, кроме точки $r=0$, в которой расположен источник.
\item Поток соленоидального векторного поля через любое поперечное сечение векторной трубки один и тот же. Это следует из того, что поток через боковую поверхность трубки равен нулю.
\end{itemize}

\subsection{Потенциальное векторное поле}

Потенциальное (или безвихревое) векторное поле в математике — векторное поле, которое можно представить как градиент некоторой скалярной функции координат. \textbf{Необходимым условием} потенциальности векторного поля в трёхмерном пространстве является равенство нулю ротора поля. Однако это условие \textbf{не является достаточным} — если рассматриваемая область пространства не является односвязной, то скалярный потенциал может быть многозначной функцией.

Пусть $\vec {v}$ — потенциальное векторное поле; оно выражается через потенциал $\phi$  как

$\vec  v=\nabla \phi$  (или в другой записи $\vec {v}=\operatorname {grad} \phi $.

Для поля сил и потенциала сил эта же формула записывается как

${\vec  F}({\vec  r},t)=-\nabla U({\vec  r},t)$,
то есть для сил потенциалом $\phi$  является $-U$.

\subsection{Условия потенциальности векторного поля}

Пусть $A = \{A_x, A_y, A_z\}$ – дифференцируемое потенциальное поле, $dr = \{dx,dy,dz\}$ – бесконечно малый вектор смещения их произвольной точки $M(x,y,z)$.

Рассмотрим скалярное произведение векторов  $A$  и  $dr$:
$$A\dot dr = \operatorname{grad} \phi \dot dr = \frac{d\phi}{dx}dx +\frac{d\phi}{dy}dy + \frac{d\phi}{dz}dz$$

Выражение в правой части этого равенства представляет собой полный дифференциал функции . Если частные производные $\frac{d\phi}{dx}, \frac{d\phi}{dy}, \frac{d\phi}{dz}$ являются непрерывными функциями, то смешанные производные $\phi$ не зависят от порядка дифференцирования:

$$\frac{d^2phi}{dxdy} = \frac{d^2phi}{dydx},\frac{d^2phi}{dzdy} = \frac{d^2phi}{dydz}, \frac{d^2phi}{dxdz} = \frac{d^2phi}{dzdx} $$

Учитывая, что частные производные от функции $\phi$ являются координатами вектора $A$, получаем следующие условия потенциальности поля $A$:
$$\frac{\partial A_y}{dx} = \frac{\partial A_x}{dy}$$
$$\frac{\partial A_z}{dx} = \frac{\partial A_x}{dz}$$
$$\frac{\partial A_y}{dz} = \frac{\partial A_z}{dy}$$

\subsection{Вычисление потенциала}

\begin{itemize}
\item Из условий потенциальности:

$$\frac{\partial A_y}{dx} = \frac{\partial A_x}{dy}$$
$$\frac{\partial A_z}{dx} = \frac{\partial A_x}{dz}$$
$$\frac{\partial A_y}{dz} = \frac{\partial A_z}{dy}$$

Отсюда получаем диффуравнения вида:

$$\frac{\partial phi}{dx} = f(x,y,z)$$
$$\frac{\partial phi}{dy} = f(x,y,z)$$
$$\frac{\partial phi}{dz} = f(x,y,z)$$

И решаем их

\item По формуле:

$$\phi(M)-\phi(M^0) = \int\limits_{M^0M}F_xdx + F_ydy + F_zdz = \int\limits_{x_0}x F_xdx+ \int\limits_{y_0}^y F_ydy+  \int\limits_{z_0}^z F_zdz$$

для потенциала векторного поля $\bar F = \{F_x, F_y, F_z\}$ в точке $M$ относительно постоянной точки $M^0$



\end{itemize}







