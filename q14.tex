\subsection{Ортогональные полиномы}
 Два полинома, заданные на интервале $\left[ {a,b} \right]$, являются ортогональными, если выполнено условие 
 $$\int\limits_a^b {p\left( x \right)q\left( x \right)w\left( x \right)dx} = 0$$ 
 где ${w\left( x \right)}$ - неотрицательная весовая функция.

Множество полиномов ${p_n}\left( x \right),\;n = 0,1,2, \ldots ,$ где $n$ - степень полинома ${p_n}\left( x \right),$ образуют систему ортогональных полиномов, если справедливо равенство $$\int\limits_a^b {{p_m}\left( x \right){p_n}\left( x \right)w\left( x \right)dx} = {c_n}{\delta _{mn}}$$ где ${c_n}$ - заданные константы, а ${\delta _{mn}}$ - символ Кронекера.
\subsection{Обобщенный ряд Фурье}
Обобщенным рядом Фурье для некоторой функции называется ее разложение в ряд на основе системы ортогональных полиномов. Любая кусочно непрерывная функция может быть представлена в виде обобщенного ряда Фурье: $$ \sum\limits_{n = 0}^\infty {{c_n}{p_n}\left( x \right)} = \begin{cases} f\left( x \right), & \text{если}\,f\left( x \right)\,\text{непрерывна} \\ \frac{{f\left( {x - 0} \right) + f\left( {x + 0} \right)}}{2}, & \text{в точке разрыва 2 рода} \end{cases}. $$ Ниже мы рассмотрим $4$ вида ортогональных полиномов: полиномы Эрмита, Лагерра, Лежандра и Чебышева.
\subsection{Полиномы Эрмита}
Полиномы Эрмита $${H_n}\left( x \right) = {\left( { - 1} \right)^n}{e^{{x^2}}}\large\frac{{{d^n}}}{{d{x^n}}}\normalsize {e^{ - {x^2}}}$$ ортогональны с весовой функцией ${e^{ - {x^2}}}$ на интервале $\left( { - \infty ,\infty } \right):$ 
$$ \int\limits_{ - \infty }^\infty {{e^{ - {x^2}}}{H_m}\left( x \right){H_n}\left( x \right)dx} = \begin{cases} 0, & m \ne n \\ {2^n}n!\sqrt \pi, & m = n \end{cases}. $$
Иногда используется альтернативное определение, в котором весовая функция равна ${e^{ - \frac{{{x^2}}}{2}}}.$ Это соглашение распространено в теории вероятностей, в частности, из-за того, что плотность нормального распределения описывается функцией $$\rho=\large\frac{1}{{\sqrt {2\pi } }}\normalsize {e^{ - \frac{{{x^2}}}{2}}}$$
\subsection{Полиномы Лагерра}
Полиномы Лагерра 
${L_n}\left( x \right) = {\large\frac{{{e^x}}}{{n!}}\normalsize} {\large\frac{{{d^n}\left( {{x^n}{e^{ - x}}} \right)}}{{d{x^n}}}\normalsize},\;n = 0,1,2,3, \ldots $ 
ортогональны с весовой функцией ${{e^{ - x}}}$ на интервале $\left( {0,\infty } \right):$ 
$$ \int\limits_0^\infty {{e^{ - x}}{L_m}\left( x \right){L_n}\left( x \right)dx} = \begin{cases} 0, & m \ne n \\ 1, & m = n \end{cases}. $$
\subsection{Полиномы Лежандра}
Полиномы Лежандра ${P_n}\left( x \right) = {\large\frac{1}{{{2^n}n!}}\normalsize} {\large\frac{{{d^n}{{\left( {{x^2} - 1} \right)}^n}}}{{d{x^n}}}\normalsize},\;n = 0,1,2,3, \ldots $ 
ортогональны на отрезке $\left[ {-1,1} \right]:$ 
$$ \int\limits_{ - 1}^1 {{P_m}\left( x \right){P_n}\left( x \right)dx} = \begin{cases} 0, & m \ne n \\ \frac{2}{{2n + 1}}, & m = n \end{cases}$$
\subsection{Полиномы Чебышева}
Полиномы Чебышева ${T_n}\left( x \right) = \cos \left( {n\arccos x} \right)$ первого рода ортогональны на отрезке $\left[ {-1,1} \right]$ с весовой функцией $\large\frac{1}{{\sqrt {1 - {x^2}} }}\normalsize :$ 
$$ \int\limits_{ - 1}^1 {\frac{{{T_m}\left( x \right){T_n}\left( x \right)}}{{\sqrt {1 - {x^2}} }}dx} = \begin{cases} 0, & m \ne n \\ \pi, & m = n = 0 \\ \frac{\pi }{2}, & m = n \ne 0 \end{cases}. $$

