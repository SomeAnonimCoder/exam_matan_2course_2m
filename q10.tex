Тригонометрический ряд Фурье — представление произвольной функции $f$ с периодом $\tau$  в виде ряда
$$f(x)=\frac{a_0}{2} + \sum^{\infin}_{n=1} (a_n \cos nx + b_n \sin nx)$$
или используя комплексную запись, в виде ряда:

$$f(x) = \sum\limits_{k=-\infty}^{+\infty} \hat{f}_k e^{ikx}$$

где

$$a_0= \frac{1}{\pi}\int\limits_{-\pi}^{\pi}f(x)dx$$
$$a_n= \frac{1}{\pi}\int\limits_{-\pi}^{\pi}f(x)\cos(nx)dx$$
$$b_n= \frac{1}{\pi}\int\limits_{-\pi}^{\pi}f(x)\sin(nx)dx$$


Тригонометрическая система ортогональна в том смысле что

$$\int\limits_{-\pi}^{\pi}cos(kx)cos(mx)=0, k,m \in N, k\ne m$$
$$\int\limits_{-\pi}^{\pi}sin(kx)sin(mx)=0, k,m \in N, k\ne m$$
$$\int\limits_{-\pi}^{\pi}cos(kx)sin(mx)=0, k,m \in N, k\ne m$$
$$\int\limits_{-\pi}^{\pi}cos^2(kx)=\pi, k \in N$$
$$\int\limits_{-\pi}^{\pi}sin^2(kx)=\pi, k \in N$$


