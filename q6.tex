Для обозначения потока векторного поля через замкнутую поверхность испльзуются выражения вида:

$$\Phi = \int\int_S A\cdot ndS$$
$$\Phi = \int\int_S A\cdot dS$$
$$\Phi = \int\int_S A_ndS$$

Поток векторного поля $А$ через замкнутую поверхность $S$ равен сумме потоков через поверхности $S_1$ и $S_2$

Если область $V$ разделить произвольным образом на $k$ элементов: $V_1, V_2,...V_k$, ограниченных поверхностям  $S_1, S_2,.., S_k$, то поток вектора $A$ через внешнюю поверхность $S$ равен сумме потоков через замкнутые поверхности $S_1,S_2,.., S_k$

$$\int\int_S A_ndS = \sum\limits_{i=1}^{k}\int\int_{S_i}A_ndS$$

\subsection{ Теорема Остроградского-Гаусса.}

Поток векторного поля $A$ через замкнутую кусочно-гладкую поверхность $S$ в направлении внешней нормали равен тройному интегралу от $div A$ по области $V$, ограниченной поверхностью $S$

$$\int\int_S A_ndS = \int\int\int_V div A dV$$

\subsection{Инвариантное определение дивергенции.}

Пусть в области $G$, ограниченной поверхностью S, определено векторное поле $\overrightarrow{a}(M)$

$$div \overrightarrow{a}(M)\cdot V(G) = \int\int_S a_ndS$$
$$div\overrightarrow{a}(M) = \frac{\int\int_S a_ndS}{V(G)}$$

$V(G)$ - объем области $G$, $M$ - некоторая точка $\in G$

Зафиксируем $M \in G$ и будем стягивать область $G$ к точке $M$ так, чтобы $M$ оставалась внутренней точкой области $G$. Тогда $V(G) \to 0$

$$div \overrightarrow{a}(M) = \lim\limits_{V(G)\to 0} \frac{\int\int_S a_ndS}{V(G)}$$

Дивергенция зависит от самого поля и не зависит от системы координат