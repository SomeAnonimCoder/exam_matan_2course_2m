\documentclass[11pt]{article}
\usepackage{amsmath}
\usepackage[russian]{babel}
\usepackage{graphicx}
\begin{document}
\subsection{Определение}
Скалярное поле (скалярная функция) на некотором конечномерном пространстве $V$ - функция, ставящая в соответствие каждой точке из некоторой области этого пространства (область определения) скаляр, то есть действительное или комплексное число. При фиксированном базисе пространства скалярное поле является функцией нескольких переменных, являющихся координатами точки.

Разница между числовой функцией нескольких переменных и скалярным полем заключается в том, что в другом базисе скалярное поле как функция координат изменяется так, что если новый набор переменных представляет ту же точку пространства в новом базисе, то значение скалярной функции не изменяется.

Например, в некотором ортонормированном базисе двумерного векторного пространства скалярная функция имеет вид $f(v)=x^{2}+2y^{2}$ в другом базисе, повернутом на 45 градусов к этому, эта же функция в новых координатах будет иметь вид $f(v)=3x^{2}+3y^{2}-2xy$.
\subsection{Линии и поверхности уровня}
Скалярное поле можно представить графически с помощью поверхностей уровня (также называемой изоповерхностями).

Поверхностью уровня скалярного поля $u=u(x,y,z)$ называется множество точек пространства, в которых функция u принимает одно и то же значение $c$, то есть поверхность уровня определяется уравнением $u(x,y,z)=c$. Изображение набора поверхностей уровня для разных $c$ дает наглядное представление о конкретном скалярном поле, для которого они построены (изображены), кроме того, представление о поверхностях уровня дает определенный дополнительный геометрический инструмент для работы со скалярным полем, который может использоваться для вычислений, доказательства теорем и т. п. Пример: эквипотенциальная поверхность.

Для поля на двумерном пространстве аналогом поверхности уровня является линии уровня. Примеры: изобата, изотерма, изогипса (линия равных высот) на географической карте и прочие изолинии.

Поверхностями уровня для скалярного поля на пространстве большей размерности являются гиперповерхности с размерностью на единицу меньшей, чем размерность пространства.

\subsection{Градиент}

Направление скорейшего возрастания поля $r=u(x,y,z)$ указывает вектор градиента, обозначаемый стандартно:
$\mathbf{grad}\space u$,
или иное обозначение:
$\nabla u$,
с компонентами:
$$\left({\frac  {\partial u}{\partial x}},\ {\frac  {\partial u}{\partial y}},\ {\frac  {\partial u}{\partial z}}\right)$$

Абсолютная величина вектора градиента $u$ есть производная $u$ по направлению скорейшего роста (скорость роста u при движении с единичной скоростью в этом направлении).

Градиент всегда перпендикулярен поверхностям уровня (в двумерном случае — линиям уровня). Исключение — особые точки поля, в которых градиент равен нулю.

\subsection{Производная по направлению}

Рассмотрим дифференцируемую функцию $f(x_{1},\ldots ,x_{n})$ от $n$ аргументов в окрестности точки $\vec {x}=(x_{1}^{0},\;\ldots ,\;x_{n}^{0})$.
 Для любого единичного вектора $\vec {e}=(e_{1},\;\ldots ,\;e_{n})$ определим производную функции
 $f$ в точке $\vec  {x}^{0}$ по направлению $\vec{e}$ следующим образом:

$$ \nabla _{\mathbf {e} }{f}(\mathbf {x} )=\lim _{h\to 0}{\frac {f({\vec {x}}{\,}^{0}+h\cdot {\vec {e}})-f({\vec {x}}{\,}^{0})}{h}}$$
Значение этого выражения показывает, как быстро меняется значение функции при сдвиге аргумента в направлении вектора $\vec{e}$.

Пусть направляющий вектор направления $e$ имеет координаты $e_{1},e_{2}\dots e_{n}$. Тогда имеет место формула:

$$ \nabla _{\mathbf {e} }{f}(\mathbf {x} )={\frac {\partial f}{\partial x_{1}}}e_{1}+\dots +{\frac {\partial f}{\partial x_{n}}}e_{n}$$

На языке векторного анализа эту формулу можно записать иначе. Производную по направлению дифференцируемой по совокупности переменных функции можно рассматривать как проекцию градиента функции на это направление, или иначе, как скалярное произведение градиента на орт направления:

$$\nabla _{\mathbf {e} }{f}(\mathbf {x} )=\nabla f\cdot {\vec {e}}$$
Отсюда следует, что в заданной точке производная по направлению принимает максимальное значение тогда, когда её направление совпадает с направлением градиента функции в данной точке.



\end{document}
