
\begin{center}
	\textbf{Поверхностный интеграл 1-го рода}
\end{center}

Пусть имеется гладкая (кусочно-гладкая) поверхность $\Sigma$ . В точках этой
поверхности задана функция $f(x,y,z)$. Интеграл вида $\iint\limits_\Sigma f(x,y,z) d\sigma$
называется поверхностным интегралом $I$ рода от функции $f(x,y,z)$ по поверхности
$\Sigma$. С физической точки зрения он представляет массу поверхности в точках
которой задана плотность $f(x,y,z)$.
Приведём основные свойства поверхностного интеграла первого рода:

\par\bigskip

\textbf{Свойство 1.} Если $u=f(x,y,z)$ непрерывна на $\Sigma$ , то $f(x,y,z)$ интегрируема.
\par\bigskip
\textbf{Свойство 2.} Линейность. Если $f(x,y,z)$, $g(x,y,z)$ интегрируемы на
$\Sigma$, то их произведение и сумма $\alpha f(x,y,z) + \beta g(x,y,z)$тоже интегрируемы на
$\Sigma$ для любых $\alpha, \beta$.
\par\bigskip
\textbf{Свойство 3.} Для неперекрывающихся гладких (кусочно-гладких) поверхностей $\Sigma_1$ и $\Sigma_2$ при интегрируемой $f(x,y,z)$ будет существовать:
$$\iint\limits_\Sigma f(x,y,z) d\sigma = \iint\limits_{\Sigma_1} f(x,y,z) d\sigma + \iint\limits_{\Sigma_2} f(x,y,z) d\sigma,$$

где $\Sigma=\Sigma_1 \cup \Sigma_2$. Здесь $\Sigma_1$ и $\Sigma_2$ называются неперекрывающимися, если $\Sigma_1 \cap \Sigma_2$ содержит конечное число кусочно-гладких кривых (может быть и пустое). Это свойство 3 называется аддитивностью интеграла $\Sigma$ по области.
\par\bigskip
\textbf{Свойство 4.}  $\iint\limits_\Sigma 1 d\sigma = \Delta\Sigma$ - площадь поверхности $\Sigma$.
\par\bigskip
\textbf{Свойство 5.} Монотонность интеграла. Если $f(x,y,z) \leq g(x,y,z)$ на и функции интегрируемы,то:

$$\iint\limits_\Sigma f(x,y,z) d\sigma \leq \iint\limits_\Sigma g(x,y,z) d\sigma $$

\textbf{Свойство 6.} Если $f(x,y,z)$ интегрируема на $\Sigma$, то $|f(x,y,z)|$ тоже интегрируема на $\Sigma$ и:

$$\left| \iint\limits_\Sigma f(x,y,z) d\sigma\right| \leq \iint\limits_\Sigma |f(x,y,z)| d\sigma. $$

\begin{center}
	\textbf{Сведение поверхностного интеграла 1-го рода к двойному}
\end{center}

\textbf{Случай 1.} Параметрически заданная поверхность. Пусть поверхность $\Sigma$
кусочно-гладкая, задана как $x=x(u,v),\, y=y(u,v),\, z=z(u,v)$, где $(u,v) \in D$ - квадрируемая область и $f(x,y,z)$ - непрерывна на $\Sigma$. Тогда существует:
$$\iint\limits_\Sigma f(x,y,z) d\sigma = \iint\limits_D f(x(u,v), y(u,v), z(u,v)) \sqrt{EG-F^2} dudv.$$
Здесь в интеграле 1-го рода коэффициенты Гаусса $E,G,F$ имеют вид:
$$ E=x_u^{'2}+y_u^{'2}+z_u^{'2}, \,\,\, G=x_v^{'2}+y_v^{'2}+z_v^{'2}, \,\,\, F=x_u^{'}x_v^{'}+y_u^{'}y_v^{'}+z_u^{'}z_v^{'}$$
и поверхностный интеграл 1-го рода существует, если существует двойной интеграл. Оба интеграла существуют, если $f(x,y,z)$ непрерывна на $\Sigma$.

\par\bigskip

\textbf{Случай 2.} Явное задание поверхности $\Sigma$ . Если $\Sigma$ - гладкая поверхность, заданная уравнением $z=z(x,y)$ при $(x,y) \in D$ - квадрируемой области, и $f(x,y,z)$ - ограниченная функция на $\Sigma$ , то интеграл 1-го рода:
$$\iint\limits_\Sigma f(x,y,z) d\sigma = \iint\limits_D f(x,y, z(x,y)) \sqrt{1+z_x^{'2}+z_y^{'2}} dxdy.$$

и существует, если существует двойной интеграл. Для гладкой $\Sigma$ и непрерывной $f(x,y,z)$, оба интеграла существуют одновременно.


В случае поверхности, заданной уравнением $x=x(y,z)$:

$$\iint\limits_\Sigma f(x,y,z) d\sigma = \iint\limits_{D_1} f(x(y,z),y,z) \sqrt{1+x_y^{'2}+x_z^{'2}} dydz,$$

где $(y,z) \in D_1$ - квадрируемая область, а также при $y=y(x,z)$ имеем:
$$\iint\limits_\Sigma f(x,y,z) d\sigma = \iint\limits_{D_2} f(x,y(x,z),z) \sqrt{1+y_x^{'2}+y_z^{'2}} dxdz,$$

для $(x,z) \in D_2$ - квадрируемой области. Отметим, что все эти формулы получаются как частные для параметрически заданной $\Sigma$.

\begin{center}
	\textbf{Поверхностный интеграл 2-го рода}
\end{center}

 Пусть $\Sigma$ - квадрируемая, гладкая двусторонняя поверхность. Фиксируем
одну из её сторон и $P(x,y,z), Q(x,y,z), R(x,y,z)$ - функции, определенные
на $\Sigma$. Тогда интеграл 
$$I=\iint\limits_\Sigma(P\cos\alpha + Q\cos\beta + R\cos\gamma) d\sigma$$
 называется поверхностным интегралом 2-го рода по выбранной стороне поверхности. Здесь
$\cos\alpha, \cos\beta, \cos\gamma$ - косинусы нормали к поверхности, направление которой
согласовано с выбранной стороной. Поверхностный интеграл 2-го рода записывают еще как 
$$I=\iint\limits_\Sigma Pdydz+Qdzdx+Rdxdy.$$ 

При переходе к другой стороне поверхности $I$ меняет знак на противоположный. Отметим, что основные
свойства поверхностного интеграла 2-го рода такие же, как у поверхностного
интеграла 1-го рода. Это линейность относительно подынтегральной функции 
и аддитивность по области при $\Sigma=\Sigma_1\cup \Sigma_2$ для неперекрывающихся поверхностей $\Sigma_1$ и $\Sigma_2$ . Также, если подынтегральная функция непрерывна, то поверхностный интеграл 2-го рода существует.

\begin{center}
	\textbf{Сведение поверхностного интеграла 2-го рода к двойному}
\end{center}

\textbf{Случай 1.} Явное задание поверхности. Пусть гладкая (или кусочногладкая) поверхность $\Sigma$ задана уравнением $z = z(x, y)$ и взята верхняя часть
этой поверхности, а $R(x, y,z)$ - ограниченная на $\Sigma$ функция. Тогда справедливо равенство:

$$\iint\limits_\Sigma R(x, y, z)dxdy = \iint\limits_D R(x, y, z(x, y))dxdy,$$

 где $D$ - проекция поверхности $\Sigma$ на плоскость $xOy$. Интеграл слева существует, если существует двойной интеграл. Если же берется нижняя сторона поверхности, то:
 
$$\iint\limits_\Sigma R(x, y,z)dxdy = - \iint\limits_D R(x, y, z(x, y))dxdy$$

Здесь нижняя и верхняя стороны поверхности $\Sigma$ отличаются противоположным направлением нормали. Отсюда и противоположные знаки в двойном интеграле. Аналогично получаем формулы:

$$\iint\limits_\Sigma P(x, y,z)dydz = \pm \iint\limits_D P(x(y,z), y, z)dydz$$

$$\iint\limits_\Sigma Q(x, y,z)dzdx = \pm \iint\limits_D Q(x, y(z,x), z)dzd[]$$
Здесь $D_1$ и $D_2$ проекции $\Sigma$ на плоскость $yOz$ и $zOx$ соответственно и оба интеграла – поверхностный и двойной существуют для непрерывных $P,Q, R$.


\par\bigskip

\textbf{Случай 2.} Параметрическое задание поверхностей. Пусть гладкая (или
кусочно-гладкая) поверхность  $\Sigma$ задана равенствами:
$x = x(u, v), y = y(u, v), z = z(u, v), (u, v) \in D$,
а $P = P(x, y, z), Q = Q(x, y, z), R = R(x, y, z)$ ограниченные на  $\Sigma$ функции.

Тогда для выбранной стороны поверхности  $\Sigma$ верно равенство:
$$\iint\limits_\Sigma Pdydz+Qdzdx+Rdxdy = \iint\limits_D (P\cos\alpha + Q\cos\beta + R\cos\gamma)\sqrt{EG-F^2} dudv,$$

где $E,G,F$ - коэффициенты Гаусса, а $\cos\alpha, \cos\beta, \cos\gamma , P,Q,R$ берутся в точке $M = (x(u, v), y(u, v), z(u, v))$. Здесь косинусы нормали имеют вид:

$$ \cos\alpha = \pm \frac{A}{\sqrt{A^2+B^2+C^2}}, \,\,\, \cos\beta = \pm \frac{B}{\sqrt{A^2+B^2+C^2}}, \,\,\, \cos\gamma = \pm \frac{C}{\sqrt{A^2+B^2+C^2}},$$

где величины:

$$A = \begin{vmatrix}
y_u^{'} & z_u^{'} \\
y_v^{'} & z_v^{'}
\end{vmatrix}, \,\,\, B = \begin{vmatrix}
z_u^{'} & x_u^{'} \\
z_v^{'} & x_v^{'}
\end{vmatrix}, \,\,\, C = \begin{vmatrix}
x_u^{'} & y_u^{'} \\
x_v^{'} & y_v^{'}
\end{vmatrix}, \,\,\, \sqrt{A^2+B^2+C^2} = \sqrt{EG+F^2}$$
 
Знаки $\pm$ в косинусах нормали соответствуют
выбранной стороне поверхности $\Sigma$ . При этом имеем значение:
$$\iint\limits_\Sigma Pdydz+Qdzdx+Rdxdy = \pm \iint\limits_D(PA+QB+RC)dudv,$$

где подинтегральные функции в двойном интеграле берутся в точке
$M = (x(u, v), y(u, v), z(u, v))$ и $\pm$ соответствует стороне поверхности.
