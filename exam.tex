\documentclass[11pt]{article}
\usepackage{amsmath}
\usepackage[russian]{babel}
\usepackage{graphicx}
\begin{document}
\tableofcontents
\section{Скалярное поле. Поверхности и линии уровня. Производная по направлению. Градиент скалярного поля.}
\section{Векторное поле. Уравнение векторных линий векторного поля. Дивергенция и ротор векторного поля.}
\section{Односторонние и двусторонние поверхности. Понятие площади поверхности. Методы вычисления площади поверхности}
\section{Поверхностные интегралы 1го и 2го рода, определение и методы вычисления}
\section{ Поток векторного поля. Определение потока, свойства и методы вычисления. Физический смысл потока}
\section{Поток векторного поля через замкнутую поверхность. Теорема Остроградского Гаусса. Инвариантное определение дивергенции}
\section{Линейный интеграл. Понятие линейного интеграла, свойства и методы вычисления. Физический смысл линейного интеграла}
\section{Циркуляция векторного поля. Теорема Стокса. Инвариантное определение ротора. Формула Грина}
\section{Потенциальные и соленоидальные векторные поля. Условия потенциальности векторного поля. Потенциал. Методы вычисления потенциала. Свойства соленоидального поля}
\section{ Тригонометрический ряд Фурье. Тригонометрическая система. На-хождение коэффициентов ряда Фурье}
\section{ Кусочно непрерывные и кусочно гладкие функции. 3 леммы (доказать одну)}
\section{Теорема Дирихле о сходимости суммы ряда Фурье. Сформулировать идею доказательства}
\section{Комплексная форма ряда Фурье}
Как известно из курса алгебры, экспонента от чисто мнимого аргумента определяется равенством $e^{i\varphi }=\cos \varphi +i\sin \varphi $.Отсюда немедленно вытекают формулы Эйлера 

$$
\cos \varphi =\frac{e^{i\varphi }+e^{-i\varphi }}{2},
\qquad
\sin \varphi =\frac{e^{i\varphi }-e^{-i\varphi }}{2i},$$

справедливые для всех вещественных чисел $\varphi $.
Предполагая, что функция f разлагается в ряд Фурье, заменим в нем синусы и косинусы по формулам Эйлера:

$$
f(x)= \frac{a_0}{2}+\sum\limits_{n=1}^\infty (a_n\cos nx
+b_n\sin nx)=$$

$$
= \frac{a_0}{2}+\sum\limits_{n=1}^\infty
\left( a_n \frac {e^{inx }+e^{-inx }}{2}
+b_n\frac{e^{inx }-e^{-inx }}{2i}
\right) =$$

$$
=\sum\limits_{n=1}^\infty\frac{a_n-ib_n}{2}e^{inx }+
\frac{a...
 ...n}{2}e^{-inx }=
\sum\limits_{n=-\infty }^{+\infty }c_ne^{inx },$$

где использованы обозначения
$$c_n = \frac{a_n-ib_n}{2} if n>0$$
$$c_n = \frac{a_0}{2} if n=0$$
$$c_n = \frac{a_{-n}-ib_{-n}}{2} if n<0$$
Вновь используя формулы Эйлера, преобразуем выражения для коэффициентов cn:

$$
c_n=
(a_n-ib_n)/2=
\frac{1}{2\pi }
\int\limits_{-\pi }^{\pi } f(x)[\cos nx -i\sin nx]\, dx=$$
$$=\frac{1}{2\pi }\int\limits_{-\pi }^{\pi }f(x) e^{-inx} dx, if n > 0$$

$$
c_0=a_0/2=
\frac{1}{2\pi }
\int\limits_{-\pi }^{\pi } f(x)\,...
 ...
\frac{1}{2\pi }
\int\limits_{-\pi }^{\pi } f(x) e^{-i0x}\, dx;$$

$$
c_n=
(a_{-n}+ib_{-n})/2=
\frac{1}{2\pi }
\int\limits_{-\pi }^{\pi } f(x)[\cos nx -\sin nx]\, dx=$$

$$
=\frac{1}{2\pi }
\int\limits_{-\pi }^{\pi } f(x) e^{-inx}\, dx,
\mbox{ если $n<0$.}$$

Итак, мы видим, что для всех значений n коэффициенты cn ищутся по одной формуле

$$
c_n=\frac{1}{2\pi }
\int\limits_{-\pi }^{\pi } f(x) e^{-inx}\, dx,
\quad
n=0,\pm 1,\pm 2, \dots .$$

При этом имеет место разложение
$$
f(x)=\sum\limits_{n=-\infty }^{+\infty }c_ne^{inx },$$

называемое комплексной формой ряда Фурье.  Оно короче и симметричнее своего вещественного аналога и поэтому чаще используется в физике.

\section{Понятие обобщенного ряда Фурье и ортогональные полиномы}
\subsection{Ортогональные полиномы}
 Два полинома, заданные на интервале $\left[ {a,b} \right]$, являются ортогональными, если выполнено условие 
 $$\int\limits_a^b {p\left( x \right)q\left( x \right)w\left( x \right)dx} = 0$$ 
 где ${w\left( x \right)}$ - неотрицательная весовая функция.

Множество полиномов ${p_n}\left( x \right),\;n = 0,1,2, \ldots ,$ где $n$ - степень полинома ${p_n}\left( x \right),$ образуют систему ортогональных полиномов, если справедливо равенство $$\int\limits_a^b {{p_m}\left( x \right){p_n}\left( x \right)w\left( x \right)dx} = {c_n}{\delta _{mn}}$$ где ${c_n}$ - заданные константы, а ${\delta _{mn}}$ - символ Кронекера.
\subsection{Обобщенный ряд Фурье}
Обобщенным рядом Фурье для некоторой функции называется ее разложение в ряд на основе системы ортогональных полиномов. Любая кусочно непрерывная функция может быть представлена в виде обобщенного ряда Фурье: $$ \sum\limits_{n = 0}^\infty {{c_n}{p_n}\left( x \right)} = \begin{cases} f\left( x \right), & \text{если}\,f\left( x \right)\,\text{непрерывна} \\ \frac{{f\left( {x - 0} \right) + f\left( {x + 0} \right)}}{2}, & \text{в точке разрыва 2 рода} \end{cases}. $$ Ниже мы рассмотрим $4$ вида ортогональных полиномов: полиномы Эрмита, Лагерра, Лежандра и Чебышева.
\subsection{Полиномы Эрмита}
Полиномы Эрмита $${H_n}\left( x \right) = {\left( { - 1} \right)^n}{e^{{x^2}}}\large\frac{{{d^n}}}{{d{x^n}}}\normalsize {e^{ - {x^2}}}$$ ортогональны с весовой функцией ${e^{ - {x^2}}}$ на интервале $\left( { - \infty ,\infty } \right):$ 
$$ \int\limits_{ - \infty }^\infty {{e^{ - {x^2}}}{H_m}\left( x \right){H_n}\left( x \right)dx} = \begin{cases} 0, & m \ne n \\ {2^n}n!\sqrt \pi, & m = n \end{cases}. $$
Иногда используется альтернативное определение, в котором весовая функция равна ${e^{ - \frac{{{x^2}}}{2}}}.$ Это соглашение распространено в теории вероятностей, в частности, из-за того, что плотность нормального распределения описывается функцией $$\rho=\large\frac{1}{{\sqrt {2\pi } }}\normalsize {e^{ - \frac{{{x^2}}}{2}}}$$
\subsection{Полиномы Лагерра}
Полиномы Лагерра 
${L_n}\left( x \right) = {\large\frac{{{e^x}}}{{n!}}\normalsize} {\large\frac{{{d^n}\left( {{x^n}{e^{ - x}}} \right)}}{{d{x^n}}}\normalsize},\;n = 0,1,2,3, \ldots $ 
ортогональны с весовой функцией ${{e^{ - x}}}$ на интервале $\left( {0,\infty } \right):$ 
$$ \int\limits_0^\infty {{e^{ - x}}{L_m}\left( x \right){L_n}\left( x \right)dx} = \begin{cases} 0, & m \ne n \\ 1, & m = n \end{cases}. $$
\subsection{Полиномы Лежандра}
Полиномы Лежандра ${P_n}\left( x \right) = {\large\frac{1}{{{2^n}n!}}\normalsize} {\large\frac{{{d^n}{{\left( {{x^2} - 1} \right)}^n}}}{{d{x^n}}}\normalsize},\;n = 0,1,2,3, \ldots $ 
ортогональны на отрезке $\left[ {-1,1} \right]:$ 
$$ \int\limits_{ - 1}^1 {{P_m}\left( x \right){P_n}\left( x \right)dx} = \begin{cases} 0, & m \ne n \\ \frac{2}{{2n + 1}}, & m = n \end{cases}$$
\subsection{Полиномы Чебышева}
Полиномы Чебышева ${T_n}\left( x \right) = \cos \left( {n\arccos x} \right)$ первого рода ортогональны на отрезке $\left[ {-1,1} \right]$ с весовой функцией $\large\frac{1}{{\sqrt {1 - {x^2}} }}\normalsize :$ 
$$ \int\limits_{ - 1}^1 {\frac{{{T_m}\left( x \right){T_n}\left( x \right)}}{{\sqrt {1 - {x^2}} }}dx} = \begin{cases} 0, & m \ne n \\ \pi, & m = n = 0 \\ \frac{\pi }{2}, & m = n \ne 0 \end{cases}. $$


\section{Интеграл Фурье и преобразование Фурье.}

Интеграл Фурье — это представление непериодической функции $f(x)$ в виде интеграла, равного непрерывной сумме гармоник, зависящих от частоты $\omega$ на интервале $[0; \infty)$.

При этом говорят, что непериодическая функция $f(x)$ имеет непрерывный спектр; частоты образующих её гармоник изменяются непрерывно. Функции $A(\omega)$ и $B(\omega)$ дают закон распределения амплитуд (и начальных фаз) в зависимости от частоты $\omega$.

Представление функции $f(x)$ на интервале $[-\infty, \infty]$:

$$\frac 1{\pi}\int\limits_0^{\infty}d\omega\int\limits_{-\infty}^{\infty}f(t)cos\omega(t-x)dt$$

$$f(x)=\int\limits_0^{\infty}[A(\omega)cos\omega x+
B(\omega)sin\omega x]d\omega$$
, где
$$A(\omega)=\frac 1{\pi}\int\limits_{-\infty}^{\infty} f(t)cos\omega tdt$$ 
$$B(\omega)=\frac 1{\pi}\int\limits_{-\infty}^{\infty}f(t)sin\omega tdt$$
Представление чётной функции $f(x)$ на интервале $(-\infty; \infty)$:

$$f(x)=\frac 2{\pi}\int\limits_0^{\infty} cos\omega xd\omega\int\limits_0^{\infty} f(t)cos\omega tdt$$
$$fчёт(x)=\int\limits_0^{\infty} A(\omega)cos\omega xd\omega$$
 где $$A(\omega)=\frac 2{\pi}\int\limits_0^{\infty} f(t)cos\omega tdt$$
Представление нечётной функции $f(x)$ на интервале $(-\infty; \infty)$:

$$f(x)=\frac 2{\pi}\int\limits_0^{\infty}sin\omega xd\omega\int\limits_0^{\infty} f(t)sin\omega tdt$$
$$f(x)=\int\limits_0^{\infty}B(\omega)sin\omega xd\omega$$
, где $$B(\omega)=\frac{2}{\pi}\int\limits_0^{\infty}f(t)sin \omega tdt$$

$$f(x)=\frac 2{\pi}\int\limits_0^{\infty}cos\omega xd\omega\int\limits_0^{\infty} f(t)cos\omega tdt$$
$$f(x)=\int\limits_0^{\infty} A(\omega)cos\omega xd\omega$$
, где $$A(\omega)=\frac2{\pi}\int\limits_0^{\infty} f(t)cos\omega tdt$$
Представление функции $f(x)$ интегралом с синусами на интервале $[0; \infty)$:

$$f(x)=\frac 2{\pi}\int\limits_0^{\infty}sin\omega xd\omega\int\limits_0^{\infty} f(t)sin\omega tdt$$
$$f(x)=\int\limits_0^{\infty} B(\omega)sin\omega xd\omega$$
, где $$B(\omega)=\frac 2{\pi}\int\limits_0^{\infty} f(t)sin\omega tdt$$


\end{document}

