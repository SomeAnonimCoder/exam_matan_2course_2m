\textbf{Кусочно-гладкая функция} — функция, определённая на множестве вещественных чисел, дифференцируемая на каждом из интервалов, составляющих область определения.

\subsection{Кусочно непрерывная функция}

$f(x)$ кусочно непрерывна на $[a,b]$ если она непрерывна на этом отрезке за исключением \textbf{конечного} числа точек(разрывов первого рода).

$x_0$ - \textbf{т. разрыва I рода}, если существуют пределы $\lim_{x\to x_0-0}f(x)$ и $\lim_{x\to x_0+0}f(x)$. Пример такой функции - $sgn(x)$. Если оба предела равны но функция в $x_0$ не определена - разрыв \textbf{устранимый}

\subsection{Кусочно гладкая функция}

$f(x)$ кусочно гладкая на $[a,b]$ если она:
\begin{itemize}
  \item кусочно непрерывна на $[a,b]$
  \item имеет в каждой точке на $[a,b]$ непрерывные производные, за исключением конечного числа точек, где $f(x)$ может не существовать или не быть непрерывной. В этих точках существуют:
  \begin{itemize}
    \item $\lim_{x\to x_0-0}f'(x)$ - левый предел производной
    \item $\lim_{x\to x_0+0}f'(x)$ - правый предел производной
  \end{itemize}
\end{itemize}

\subsection{Лемма о правой производной не существующей в точке функции}

Пусть $f(x)$ дифференцируема в правой полуокрестности точки $x_0$ - $[x_0, x_1]$ и пусть существует 
$$\lim _{x\to x_0+0}f'(x) = f'(x_0+0)$$
тогда 
\begin{itemize}
\item $\exists \lim _{x\to x_0+0}f(x) = f(x_0+0)$
\item $\exists \lim _{\chi \to 0} \frac{f(x_0+\chi)-f(x_0+0)}{\chi} = f'(x_0+0)$
\end{itemize}

Смысл человеческими словами:

Если $f(x)$ не существует в $x_0$, но существует правый предел ее производной, то существует и правый предел самой этой функции. Так же существует и правая производная функции в точке, равная пределу производной справа.

\subsection{Лемма об аппроксимации}

Пусть $f(x)$ непрерывна на $[a,b]$, тогда $\forall \varepsilon>0 \exists e(x): \forall x \in [a,b] |f(x)-e(x)|<\varepsilon, f(a)=e(a), f(b)=e(b)$, $e$ - кусочно гладкая.

По-человечески: любую непрерывную на $[a,b]$ функцию можно аппроксимировать кусочно гладкой функцией так что отличаться они будут на сколь угодно малую величину в каждой из точек отрезка, а на концах - в точках $a$ и $b$ - будут совпадать.

Доказательство:

$f(x)$ - непрерывна на $[a,b]$, значит $\forall \varepsilon >0 \exists \delta >0: \forall x, \forall x' \in [a,b] |x-x'|<\delta \Rightarrow |f(x)-f(x')|<\frac{\varepsilon}2$

По-человечески: разбиваем наш отрезок на очень маленькие отрезки, части функции соответствующие каждой части отрезка заменяем на прямые, чьи значения на краях "кусочков" совпадают с функцией. Полученная функция является кусочно гладкой - она линейна за исключением конечного числа точек. Ну и совпадение на концах отрезка очевидно.



