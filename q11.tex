\textbf{Кусочно-гладкая функция} — функция, определённая на множестве вещественных чисел, дифференцируемая на каждом из интервалов, составляющих область определения.

\subsection{Кусочно непрерывная функция}

$f(x)$ кусочно непрерывна на $[a,b]$ если она непрерывна на этом отрезке за исключением \textbf{конечного} числа точек(разрывов первого рода).

$x_0$ - \textbf{т. разрыва I рода}, если существуют пределы $\lim_{x\to x_0-0}f(x)$ и $\lim_{x\to x_0+0}f(x)$. Пример такой функции - $sgn(x)$. Если оба предела равны но функция в $x_0$ не определена - разрыв \textbf{устранимый}

\subsection{Кусочно гладкая функция}

$f(x)$ кусочно гладкая на $[a,b]$ если она:
\begin{itemize}
  \item кусочно непрерывна на $[a,b]$
  \item имеет в каждой точке на $[a,b]$ непрерывные производные, за исключением конечного числа точек, где $f(x)$ может не существовать или не быть непрерывной. В этих точках существуют:
  \begin{itemize}
    \item $\lim_{x\to x_0-0}f'(x)$ - левый предел производной
    \item $\lim_{x\to x_0+0}f'(x)$ - правый предел производной
  \end{itemize}
\end{itemize}





