Как известно из курса алгебры, экспонента от чисто мнимого аргумента определяется равенством $e^{i\varphi }=\cos \varphi +i\sin \varphi $.Отсюда немедленно вытекают формулы Эйлера 

$$
\cos \varphi =\frac{e^{i\varphi }+e^{-i\varphi }}{2},
\qquad
\sin \varphi =\frac{e^{i\varphi }-e^{-i\varphi }}{2i},$$

справедливые для всех вещественных чисел $\varphi $.
Предполагая, что функция f разлагается в ряд Фурье, заменим в нем синусы и косинусы по формулам Эйлера:

$$
f(x)= \frac{a_0}{2}+\sum\limits_{n=1}^\infty (a_n\cos nx
+b_n\sin nx)=$$

$$
= \frac{a_0}{2}+\sum\limits_{n=1}^\infty
\left( a_n \frac {e^{inx }+e^{-inx }}{2}
+b_n\frac{e^{inx }-e^{-inx }}{2i}
\right) =$$

$$
=\sum\limits_{n=1}^\infty\frac{a_n-ib_n}{2}e^{inx }+
\frac{a...
 ...n}{2}e^{-inx }=
\sum\limits_{n=-\infty }^{+\infty }c_ne^{inx },$$

где использованы обозначения
$$c_n = \frac{a_n-ib_n}{2} if n>0$$
$$c_n = \frac{a_0}{2} if n=0$$
$$c_n = \frac{a_{-n}-ib_{-n}}{2} if n<0$$
Вновь используя формулы Эйлера, преобразуем выражения для коэффициентов cn:

$$
c_n=
(a_n-ib_n)/2=
\frac{1}{2\pi }
\int\limits_{-\pi }^{\pi } f(x)[\cos nx -i\sin nx]\, dx=$$
$$=\frac{1}{2\pi }\int\limits_{-\pi }^{\pi }f(x) e^{-inx} dx, if n > 0$$

$$
c_0=a_0/2=
\frac{1}{2\pi }
\int\limits_{-\pi }^{\pi } f(x)\,...
 ...
\frac{1}{2\pi }
\int\limits_{-\pi }^{\pi } f(x) e^{-i0x}\, dx;$$

$$
c_n=
(a_{-n}+ib_{-n})/2=
\frac{1}{2\pi }
\int\limits_{-\pi }^{\pi } f(x)[\cos nx -\sin nx]\, dx=$$

$$
=\frac{1}{2\pi }
\int\limits_{-\pi }^{\pi } f(x) e^{-inx}\, dx,
\mbox{ если $n<0$.}$$

Итак, мы видим, что для всех значений n коэффициенты cn ищутся по одной формуле

$$
c_n=\frac{1}{2\pi }
\int\limits_{-\pi }^{\pi } f(x) e^{-inx}\, dx,
\quad
n=0,\pm 1,\pm 2, \dots .$$

При этом имеет место разложение
$$
f(x)=\sum\limits_{n=-\infty }^{+\infty }c_ne^{inx },$$

называемое комплексной формой ряда Фурье.  Оно короче и симметричнее своего вещественного аналога и поэтому чаще используется в физике.
