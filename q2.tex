
\begin{center}
\textbf{Векторное поле.}
\end{center}

Векторным полем называется часть пространства (или все пространство), в каждой точке $M$ которого задано какое-либо физическое явление, характеризуемое векторной величиной $\vec{a} = \vec{a}(M)$.

Если в пространстве введена декартова прямоугольная система координат, то задание вектор - функции поля $\vec{a}(M)$ cводится к заданию трех скалярных функций:

$$\vec{a}(M)=a_x(x,y,z)\vec{i}+ a_y(x,y,z)\vec{j} + a_z(x,y,z)\vec{k}$$

Простейшими геометрическими характеристиками векторных полей являются векторные линии и векторные трубки.

\begin{center}
\textbf{Уравнение векторных линий векторного поля.}
\end{center}

Векторными линиями поля $\vec{a} = \vec{a}(M)$ называются линии (кривые), в каждой точке $M$ которых направление касательной совпадает с направлением поля в этой точке.

Векторной трубкой называется поверхность, образованная векторными линиями, проходящими через точки некоторой лежащей в поле замкнутой кривой, не совпадающей (хотя бы и частично) с какой – либо векторной линией.

Для векторного поля $\vec{a}(M)=a_x(x,y,z)\vec{i}+ a_y(x,y,z)\vec{j} + a_z(x,y,z)\vec{k}$ векторные линии описываются системой дифференциальных уравнений:

$$\frac{dx}{a_x(x,y,z)}=\frac{dy}{a_y(x,y,z)}=\frac{dz}{a_z(x,y,z)}$$

Если векторное поле плоское, то при подходящем выборе системы
координат будут выполнены условия $a_x(x,y,z)=a_x(x,y), a_y(x,y,z)=a_y(x,y), a_z(x,y,z)=0$. Тогда будем иметь уравнение:


$$\frac{dy}{dx} = \frac{a_y(x,y)}{a_x(x,y)}, z=const$$	


Известно, что если в некоторой области в трехмерном пространстве
функции $a_x(x,y,z), a_y(x,y,z), a_z(x,y,z)$ непрерывны, имеют непрерывные частные производные первого
порядка и не обращаются в 0, то через каждую точку этой области проходит единственная векторная линия. Это означает, что векторные
линии не пересекаются и не касаются друг друга.

\begin{center}
\textbf{ Дивергенция и ротор векторного поля.}
\end{center}

Дивергенция характеризует отнесенную к единице объема мощность потока векторного поля, “исходящего” из точки $M$. В декартовой системе координат дивергенция вычисляется по формуле:

$$div \vec{a} = (\vec{\nabla}\cdot \vec{a})= \frac{dx}{a_x(x,y,z)}+\frac{dy}{a_y(x,y,z)}+\frac{dz}{a_z(x,y,z)}$$

Свойства дивергенции. Пусть $\vec{a}$ и $\vec{b}$ - векторные поля, $u$ - скалярная функция. Тогда:
\par\bigskip
1) $div(\vec{a}+\vec{b})=div\vec{a}+div\vec{b}$

2) $div(u\cdot\vec{a})=u\, div\vec{a} + (\vec{a}\cdot grad u)$
\par\bigskip

В трехмерном пространстве $rot\vec{a}(M)$ выражается следующим образом:

$$rot\vec{a} = [\vec{\nabla}\times \vec{a}]= \begin{vmatrix} \vec{i} & \vec{j}
& \vec{k}\\
\frac{\partial}{\partial x} & \frac{\partial}{\partial y} & \frac{\partial}{\partial z}\\
a_x & a_y & a_z \end{vmatrix} = \left(\frac{\partial a_z}{\partial y}-\frac{\partial a_y}{\partial z}\right)\vec{i}+
\left(\frac{\partial a_x}{\partial z}-\frac{\partial a_z}{\partial x}\right)\vec{j}+\left(\frac{\partial a_y}{\partial x}-\frac{\partial a_x}{\partial y}\right)\vec{k}$$

Свойства ротора (вихря). Пусть $\vec{a}$ и $\vec{b}$ - векторные поля, $u$ - скалярная функция. Тогда:
\par\bigskip

1) $rot(\lambda\vec{a} + \mu\vec{b}) = \lambda\cdot rot\vec{a} + \mu\cdot rot\vec{b}$

2) $rot(u \cdot \vec{a})=[grad u \times \vec{a}] + u\cdot rot\vec{a}$

